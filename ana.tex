
\documentclass[a6paper,11pt,print
,grid=both
%,toc
]{kartei}
\usepackage[naustrian]{babel}
%\usepackage{lmodern} 					% Latin Modern
%\usepackage{concrete} 					% Computer Concrete, Microtype deaktivieren

\usepackage{beton}
\usepackage{eulervm}					% AMS Euler, Microtype deaktivieren

%\usepackage{concmath}
%\usepackage{times}
\usepackage[utf8]{inputenc} 				% UTF-Codierung
\usepackage[T1]{fontenc}
%\usepackage{ulem}						% Unterstreichen, Durchstreichen
\usepackage{amsmath} 					% Mathemakros
\usepackage{amsfonts}
\usepackage{enumitem}
\usepackage{import}



\DeclareMathOperator\arsinh{arsinh}
\DeclareMathOperator\artanh{artanh}
\DeclareMathOperator\arcosh{arcosh}


\begin{document}

%%
%%Anmerkungen /subsections bei themen, zb absolute konvergenz kriterien.

{\Large
	
	\section*{Analysis I}
	\setcounter{section}{1}
	
	\subsection{Einleitung}
\begin{karte}{Geometrische Summe}
	Für alle reelen Zahlen \(q \in \mathbb{R},\ q\neq1\), und alle \(n \in \mathbb{N}_0\) gilt
	\begin{align}
		\sum_{k=0}^{n} q^k = \frac{1-q^{n+1}}{1-q}
	\end{align}
\end{karte}

\begin{karte}{Binomialkoeffizient}
	Für \(n\) und \(k \in \mathbb{N}_0\), \( k \leq n\) ist der \emph{Binomialkoeffizient} \(\binom{n}{k}\) definiert durch
	\begin{align}
		\binom{n}{k} := \frac{n!}{k!(n-k)!}
	\end{align}
\end{karte}

\begin{karte}{Wie lautet der Binomische Lehrsatz?}
	Für alle x, y \(\in \mathbb{R}\) und \(n \in \mathbb{N}\) gilt:
	\begin{align}
		{(x + y)}^n = \sum_{k=0}^{n}\binom{n}{k} x^{n-k}y^k=\sum_{k=0}^{n}\binom{n}{k} x^{k}y^{n-k}
	\end{align}
\end{karte}

\subsection{Mengen und Abbildungen}


\begin{karte}{Injektivität, Surjektivität, Bijektivität}
	{\large
		Eine Abbildung \(f\) heißt \emph{injektiv}, wenn für alle \(a_1,a_2 \in A\) gilt:
		\begin{align}
			a_1 \neq a_2 \implies f(a_1) \neq f(a_2),
		\end{align}
		d.h.\ zu jedem $b\in B$ gibt es \emph{höchstens} ein $a\in A$ mit\\ $b=f(a)$.

		Eine Abbildung \(f\) heißt \emph{surjektiv}, wenn jedes Element von \(B\)  als Bild eines Elements von $A$ auftritt:
		\begin{align}
			\forall \ b \in B: \exists a\in A \ mit \ b=f(a)
		\end{align}
		d.h.\ zu jedem \(b\in B\) gibt es \emph{mindestens} ein \(a\in A\) mit\\ \(b=f(a)\).

		Eine Abbildung \(f\), die sowohl injektiv als auch surjektiv ist, heißt \emph{bijektiv}. Anders ausgedrückt: Zu jedem \(b \in B\) gibt es \emph{genau ein} \(a \in A\) mit \(b=f(a)\)
	}
\end{karte}

	\subsection{Die reelen Zahlen}

\begin{karte}{Dreiecksungleichungen}
	Für alle \(x,y \in \mathbb{R}\) gilt.
	\begin{enumerate}[label=\(\triangleright \)]
		\item \(\lvert xy    \rvert =      		\lvert x\rvert    \lvert y\rvert  \)
		\item \(\lvert x\pm y\rvert \leq\  		\lvert x\rvert +  \lvert y\rvert  \)  \dotfill \emph{Dreiecksungleichung}
		\item \(\lvert x\pm y\rvert \geq\ \lvert\lvert x\rvert -  \lvert y\rvert\rvert \)  \dotfill \emph{inverse Dreiecksungleichung}
		      
		      
	\end{enumerate}
\end{karte}

\begin{karte}{Typen von Punken}
	{\large
		Sei \(A \subseteq \mathbb{R}\). Ein Punkt \(x \in \mathbb{R}\) heißt.
		\begin{enumerate}[label=$\triangleright$]
			\item \emph{innerer Punkt} von A, wenn ein \(\varepsilon > 0\) existiert, so dass \(K(x,\varepsilon) 		\subseteq A\);
			\item \emph{äußerer Punkt} bezüglich \(A\), wenn \(x\) innerer Punkt von \(A^c\) ist;
			\item \emph{Randpunkt} von \(A\), wenn  jede \(\varepsilon \)-Umgebung \(K(x,\varepsilon)\) einen Punkt von \(A\) und einen Punkt von \(A^c\) enthält;
			\item \emph{Häufungspunkt} von \(A\), wenn jede \(\varepsilon \)-Umgebung von \( x \) \emph{unendlich viele} Punkte von \(A\) enthält. Gleichbedeutend damit ist, dass jede punktierte \(\varepsilon \)-Umgebung von \(x\) mintestens einen Punkt von \(A\) enthält.
			\item \emph{isolierter Punkt} von \(A\), wenn \(x \in A \) und  wenn es eine punktierte \(\varepsilon \)-Umgebung \(K_{r}(x,\varepsilon)\) gibt, so dass \(K_{r}(x,\varepsilon) \cap A = \emptyset \).
			      
		\end{enumerate}
	}
	
\end{karte}


\begin{karte}{Wie lautet der Satz von Bolzano-Weierstraß?}
	Jede \emph{beschränkte, unendliche} Teilmenge von \(\mathbb{R}\) besitzt mindestens einen Häufungspunkt.
\end{karte}
\subsection{Zahlenfolgen}

	\subsection{Unendliche Reihen}
\begin{karte}{Geometrische Reihe}
	\begin{align}
		\sum_{k=0}^{\infty} =\frac{1}{1-q}\ \text{für} \ \lvert q\rvert < 1 
	\end{align}
\end{karte}

\begin{karte}{Formulieren Sie das Leibnitz-Kriterium für die Konvergenz von alternierenden Reihen.}
	{\large Eine Reihe der Bauart
		\begin{align}
			\sum_{k=0}^{\infty} {(-1)}^{k} a_k, \ mit \ \ a_k > 0\ \text{für }k=0,1,2,\dots 
		\end{align}
		heißt \emph{alternierende Reihe.}}\vspace{5mm}\par
	Falls die Folge \( \{ a_k \} \) monoton gegen 0 konvergiert, dann ist die alternierende Reihe
	\begin{align}
		\sum_{k=0}^{\infty}{(-1)}^k a_k 
	\end{align}
	konvergent.
\end{karte}

\begin{karte}{Was bedeutet absolute Konvergenz?}
	Eine Reihe \(\sum_{k=1}^{\infty}a_k\) heißt \emph{absolut konvergent}, wenn die Reihe \(\sum_{k=1}^{\infty}\lvert a_k \rvert \) konvergiert. \vspace{5mm}\par
	Jede absolut konvergente Reihe ist konvergent.
\end{karte}

\begin{karte}{Konvergenzkriterien: Majoranten- und Minorantenkriterium}
	\begin{enumerate}[label=\textbullet]
		
		\item Sei \(\sum_{k=1}^{\infty}a_k\) eine Reihe mit positiven Glieder, also \\ \( a_k > 0\). Falls ein \(M \in\ \mathbb{N}\) existiert, so dass für alle \(k \geq\ M\) gilt \(\lvert c_k\rvert \leq\ a_k\), dann heißt \(\sum_{k=1}^{\infty}a_k\) \emph{Majorante} der Reihe \(\sum_{k=1}^{\infty}c_k\)
		      \begin{enumerate}[label=\(\triangleright \)]
		      	\item Eine Reihe, die eine kovergente Majorante besitzt, ist absolut konvergent.
		      \end{enumerate}
		\item Sei \(\sum_{k=1}^{\infty}b_k\) eine Reihe mit positiven Glieder, also \\ \( b_k > 0\). Falls ein \(M \in \mathbb{N}\) existiert, so dass für alle \(k \geq M\) gilt \(\lvert c_k\rvert \geq a_k\), dann heißt \(\sum_{k=1}^{\infty}b_k\) \emph{Minorante} der Reihe \(\sum_{k=1}^{\infty}c_k\)
		      \begin{enumerate}[label=\(\triangleright \)]
		      	\item Eine Reihe, die eine divergente Minorante besitzt, kann nicht absolut konvergieren.
		      \end{enumerate}
		      
	\end{enumerate}
\end{karte}

\begin{karte}{Quotientenkriterium für absolute Konvergenz einer Reihe}
	Falls für die Reihe \(\sum_{k=1}^{\infty}a_k\) ab einem gewissen Index \(N\), also für alle \( k \geq N\) gilt
	\begin{enumerate}[label=\(\triangleright \)]
		
		\item \(\lvert \frac{a_{k+1}}{a_k}\rvert \leq q < 1\), dann ist die Reihe absolut konvergent.
		      
		\item \(\lvert \frac{a_{k+1}}{a_k}\rvert \geq  1\), dann ist die Reihe divergent.
		      
		\item \(\lvert \frac{a_{k+1}}{a_k}\rvert \leq  1\), jedoch nicht \(<1\), dann ist keine allgemeine Aussage möglich.
	\end{enumerate}
	
\end{karte}

\begin{karte}{Quotientenkriterium (Grenzwertformulierung)}
	Falls der Grenzwert
	\begin{align}
		\lim_{k\to\infty} \lvert \frac{a_{k+1}}{a_k}\rvert =: r 
	\end{align}
	existiert, dann gilt:
	
	\begin{enumerate}[label=\(\triangleright \)]
		
		\item für \(r<1\) ist die Reihe absolut konvergent,
		\item für \(r>1\) ist sie divergent,
		\item für \(r=1\) ist keine Aussage möglich.
	\end{enumerate}
	
	
\end{karte}

\begin{karte}{Wurzelkriterium für absolute Konvergenz einer Reihe}
	Falls für eine Reihe \(\sum_{k=1}^{\infty}a_k\) ab einem gewissen Index, also für  alle \(k \geq N\) gilt:
	
	\begin{enumerate}[label=\(\triangleright \)]
		
		\item \(\sqrt[k]{\lvert a_k\rvert} \leq q < 1\), dann ist die Reihe absolut konvergent.
		      
		\item \(\sqrt[k]{\lvert a_k\rvert} \geq  1\), dann ist die Reihe divergent.
		      
		\item \(\sqrt[k]{\lvert a_k\rvert} \leq  1\), jedoch nicht \(\leq q <1\), dann ist keine  Aussage möglich.
	\end{enumerate}
	
\end{karte}

\begin{karte}{Wurzelkriterium (Grenzwertformulierung)}
	Falls der Grenzwert
	\begin{align}
		\lim_{k\to\infty} \sqrt[k]{\lvert a_k\rvert} = r 
	\end{align}
	existiert, dann gilt:
	\begin{enumerate}[label=\(\triangleright \)]
		\item für \(r < 1\) ist die Reihe absolut konvergert,
		\item für \(r > 1\) ist sie divergent,
		\item für \(r = 1\) ist keine Aussage möglich.
	\end{enumerate}
\end{karte}

\begin{karte}{Reihendarstellung für \(e\), allgemeine Exponentialreihe}
	Die Eulersche Zahl
	\begin{align}
		e=\lim_{n\to\infty}{\left(1+\frac{1}{n}\right)}^n 
	\end{align}
	besitzt die Darstellung als unendliche Reihe:
	\begin{align}
		e=\sum_{k=0}^{\infty}\frac{1}{k!} 
	\end{align}
	allgemeine Exponentialreihe:
	\begin{align}
		\lim_{n\to\infty}{\left(1+\frac{x}{n}\right)}^n=\sum_{k=0}^{\infty}\frac{x^n}{k!}=1+x+\frac{x^2}{2!}+\frac{x^3}{3!}+\cdots=:e^x 
	\end{align}
\end{karte}

	\subsection{Reelle Funktionen}
\begin{karte}{Stetigkeit (Limes-Definition)}
	Sei \(A\subseteq\mathbb{R}\) offen. Eine Funktion \(f:D\to\mathbb{R}\) heißt \emph{stetig an der Stelle} \(c \in D\), wenn für jede konvergente Folge \(\{ x_{n} \} \) in \(D\) mit \(\displaystyle\lim_{n\to\infty} x_n = c\) gilt:
	\begin{align}
		\lim_{n\to\infty} f(x_n)=f(c),\ \ kurz:\ \lim_{x\to c}f(x)=f(c)
	\end{align}
\end{karte}

\begin{karte}{Stetigkeit (\(\varepsilon\text{-}\delta \)-Definition)}
	Sei \(A\subseteq\mathbb{R}\) offen. Eine Funktion \(f:D\to\mathbb{R}\) heißt \emph{stetig an der Stelle} \(c \in D\), wenn zu jeder reelen Zahl \(\varepsilon > 0\) eine reele Zahl \(\delta=\delta(\varepsilon)>0\) existiert, so dass für alle x mit \(|c-x|<\delta \) gilt
	\begin{align}
		|f(c)-f(x)|<\varepsilon
	\end{align}

\end{karte}

\begin{karte}{Formulieren Sie die Definition der gleichmäßigen Stetigkeit.}
	Eine Funktion $f$ heißt \emph{gleichmäßig stetig} auf dem Intervall $I$, wenn zu jedem $\varepsilon > 0$ ein $\delta = \delta(\varepsilon) > 0$ existiert, so dass für alle $x_1,x_2 \in I$ mit $\lvert x_1 - x_2 \rvert < \delta $ gilt
	\begin{align}
		\lvert f(x_1) - f(x_2) \rvert<\varepsilon
	\end{align}
	{\large
		Beachte: In der Definition der gleichmäßigen Stetigkeit darf $\delta$ nur von $\varepsilon$ abhängen, aber nicht von der Stelle $x$ (wie in der Definition der Stetigkeit an einer festen Stelle $c$)
	}
\end{karte}


\begin{karte}{Lipschitz-Stetigkeit einer Funktion \(f\).}%: $\mathbb{R} \to\mathbb{R}$.}
	Eine Funktion \(f\) heißt \emph{Lipschitz-stetig} auf dem Intervall \(I\), wenn es eine Konstante \(L\geq0\) gibt, so dass gilt
	\begin{align}
		\lvert f(x_1) - f(x_2) \rvert\leq L\ \lvert x_1 - x_2 \rvert\  \forall \  x_1,x_2\in I
	\end{align}
	{\large Man beachte, dass die Lipschitzkonstante von dem betrachteten Intervall abhängt. Viele Funktionen (z.\,B. Polynome) sind Lipschitz-stetig auf jedem beschränkten Intervall, aber nicht auf ganz $\mathbb{R}$\par}
	\vspace{5mm}
	\emph{Lipschitz-Stetigkeit} impliziert \emph{gleichmäßige Stetigkeit.}

\end{karte}

\subsection{Rationale Funktionen}

\begin{karte}{Fundamentalsatz der Algebra}
	Jedes Polynom vom Grad \(\geq 1\) hat mindesten eine (reelle oder komplexe) Nullstelle
\end{karte}

\begin{karte}{allgemeine Lösungsformel}
	\begin{align}
		x_{1,2}=\frac{-b\pm\sqrt{b^2-4ac}}{2a}
	\end{align}
\end{karte}

	\subsection{Elementare Funktionen}



\begin{karte}{(Natürliche und allgemeine) Exponentialfunktion und Logarithmus}
	{\large
		\begin{align}
			e^{x+y}         & =  e^{x}e^{y},    & e^{x\cdot y} & ={(e^{x})}^{y}, & e^{-x}           & = \frac{1}{e^{x}} \\
			\ln{(x\cdot y)} & =\ln{x} + \ln{y}, & \ln{x^{k}}   & =k\ln{x},       & \ln{\frac{1}{x}} & = -\ln{x}         \\
			a^{x+y} &=  a^{x}a^{y},				& a^{x\cdot y}&={(a^{x})}^{y}\\
			&&	a^x&={\left(e^{\ln{a}}\right)}^x=e^{x\ln{a}}\\
			&&\log_{a}x&=\frac{\ln{x}}{\ln{a}},\ 0<x<\infty
		\end{align}
	}
\end{karte}

\begin{karte}{Pythagoräische Identität, Sinussatz, Additionstheoreme}
	\begin{enumerate}[label=\(\triangleright \)]
		\item \(\sin^{2}\alpha+\cos^{2}\alpha = 1\)\dotfill \emph{Pythagoräische Identität}
		\item \(\frac{\sin{\alpha}}{a}=\frac{\sin{\beta}}{b}=\frac{\sin{\gamma}}{c} \)  \dotfill \emph{Sinussatz}
		\item \(c^2=a^2+b^2-2ab\cos{\gamma} \)  \dotfill \emph{Cosinussatz}
		\item \(\sin{(\alpha \pm \beta)} 	=	\sin{\alpha}\cos{\beta}	\pm	\cos{\alpha}\sin{\beta}\)
		\item \(\cos{(\alpha \pm \beta)} 	=	\cos{\alpha}\cos{\beta}	\mp	\sin{\alpha}\sin{\beta}\)
	\end{enumerate}
\end{karte}

\subsection{Differentialrechung}

\begin{karte}{Differenzierbarkeit}
	Sei \(f:(a,b)\to\mathbb{R}\) und \(x\in(a,b)\). Wenn \(\displaystyle\lim_{h\to0}\frac{f(x+h)-f(x)}{h}\) existiert, dann heißt \(f\) \emph{differenzierbar} an der Stelle \(x\). Man schreibt
	\begin{align}
		f'(x):=\lim_{h\to0}\frac{f(x+h)-f(x)}{h}
	\end{align}
	und bezeichnet \(f'(x)\) als die \emph{1. Ableitung} oder \emph{Differentialquotient} von \(f\) an der Stelle \(x\).
\end{karte}

\begin{karte}{Ableitung der Umkehrfunktion}
	Sei \( f(x) \) in \((a,b)\) differenzierbar und streng monoton, weiters sei \(f'(x)\neq 0, x\in(a,b)\). Dann exisitiert die Umkehrfunktion \(f^{-1}\) und ist differenzierbar. Es gilt
	\begin{align}
		(f^{-1}(y))'=\frac{1}{f'(f^{-1}(y))}
	\end{align}
\end{karte}

\begin{karte}{Ableitungen von \(\ln{x}\), \(\log_{a}{x}\) und den zyklometrischen Funktionen}
	\begin{align}
		\frac{d}{dx}\ln{x}     & =\frac{1}{x}\, ,            &   & \frac{d}{dx}\log_a x = \frac{1}{\ln{a}}\frac{1}{x} \\
		\frac{d}{dx}\arcsin{x} & =\frac{1}{\sqrt{1-x^2}}\, , &   & x\in(-1,1)                                         \\
		\frac{d}{dx}\arccos{x} & =-\frac{1}{\sqrt{1-x^2}}\,, &   & x\in(-1,1)                                         \\
		\frac{d}{dx}\arctan{x} & =\frac{1}{1+x^2}\, ,        &   & x\in\mathbb{R}
	\end{align}
\end{karte}

\begin{karte}{Ableitung von \(\arsinh{x}\), \(\arcosh{x}\) und \(\artanh{x}\)}
	\begin{align}
		\frac{d}{dx}\arsinh{x} & =\frac{1}{\sqrt{x^2+1}}\, , & x & \in\mathbb{R} \\
		\frac{d}{dx}\arcosh{x} & =\pm\frac{1}{x^2-1}\, ,     & x & >1            \\
		\frac{d}{dx}\artanh{x} & =\frac{1}{1-x^2}\, ,        & x & \in(-1,1)
	\end{align}
\end{karte}

\begin{karte}{Wie lautet der 1. Mittelwertsatz der Differentialrechnung?}

	Sei \(f(x)\) stetig auf \([a,b]\) und differenzierbar auf \((a,b)\). Dann existiert ein \(\xi \in (a,b)\) mit
	\begin{align}
		f'(\xi)=\frac{f(b)-f(a)}{b-a}
	\end{align}
	{\large
		\emph{Geometrische Interpretation:} Es gibt mindestens einen Punkt \(\xi \in (a,b)\), an dem die Steigung der Tangente an den Graphen von $f$ gleich der Steigung der Geraden durch die Punkte \((a, f(a))\) und \((b, f(b))\) ist. Siehe Abb. 9.4 im Skript.
	}
\end{karte}



\begin{karte}{Wie lautet der 2. Mittelwertsatz der Differentialrechnung?}
	Seien \(f, g\) stetig auf \([a,b]\) und differenzierbar auf \(a,b\). Dann existiert \(\xi\in(a,b)\) mit
	\begin{align}
		f'(\xi)(g(b)-g(a))=g'(\xi)(f(b)-f(a))
	\end{align}
	(Der Speizialfall \(g(x)=x\) ergibt wiederum den 1. Mittelwertsatz.)
\end{karte}



\begin{karte}{Wie lautet der Satz von Rolle?}
	Sei \(f(x)\) stetig auf \([a,b]\), differenzierbar auf \((a,b)\) und gilt \(f(a)=f(b)\). Dann existiert ein \(\xi \in (a,b) \) mit \(f'(\xi)=0\)
\end{karte}

\begin{karte}{Maximalbetrag der Ableitung als Lipschitzkonstante}
	Sei \(f\) stetig differenzierbar auf \([a,b]\). Dann ist \(f\) Lipschitz-stetig auf \([a,b]\), d.h.\ es gilt
	\begin{align}
		\lvert f(x_1)-f(x_2)\rvert \leq L\lvert x_1 - x_2\rvert \  \forall\  x_1,x_2 \in [a,b]
	\end{align}
	mit \(L=\displaystyle\max_{x\in[a,b]}\lvert f'(x)\rvert \) als kleinstmöglicher Lipschitzkonstante.

\end{karte}

\begin{karte}{Regel von de l'Hospital für \(0/0\)}
	Die Funktionen \(f,g:[c,c+\varepsilon]\to\mathbb{R}\) seien stetig und auf \((c,c+\varepsilon)\) differenzierbar. Es gelte
	\begin{align}
		  & f(c)=g(c)=0                              & und\ \ &g'(x)\neq 0, x\in(c,c+\varepsilon) &                                    \\
		\intertext{%
		Falls der Grenzwert}
		  & \lim_{x\to c+}\frac{f'(x)}{g'(x)}=\gamma &        &(\gamma=\pm\infty\ \ zugelassen!)  \\
		\intertext{%
		existiert, dann gilt auch}
		  & \lim_{x\to c+}\frac{f(x)}{g(x)}=\gamma   &        &
	\end{align}

\end{karte}

\begin{karte}{Wie lautet die Leibnizsche Produktregel}
	\begin{align}
		{(fg)}^{(n)}=\sum_{k=0}^{n} \binom{n}{k}f^{(k)}g^{(n-k)}
	\end{align}
\end{karte}

	\subsection{Verhalten reeller Funktionen}

\begin{karte}{Wie lautet der Satz von Taylor?}
	Sei $f:[a,b] \to \mathbb{R}$ eine $(n+1)mal$ stetig differenzierbare Funktion. Dann gilt für alle $x_0 \in [a,b]$ und $x=x_0 + h \in [a,b]$ die Taylorsche Formel,
	\begin{multline}
		f(x)=f(x_0+h)=\\ f(x_o) + \frac{f'(x_0)}{1!}h + \frac{f''(x_0)}{2!}h^2 +  \cdots+\frac{f^{(n)}(x_0)}{n!}h^n + R_{n+1}(x)
	\end{multline}
	mit dem Restglied der Ordnung $n+1$.\vspace{2.5mm}
	\begin{align}
		R_{n+1}(x) = \frac{f^{(n+1)}(x_0 + \vartheta\ h)}{(n+1)!}h^{(n+1)} \ \ mit \ \vartheta \in [0,1]
	\end{align}

\end{karte}

\begin{karte}{Lokale Extrema}
	Sei \(f:[a,b]\to\mathbb{R}\) stetig. \(x_0 \in (a,b)\) heißt \emph{lokales Minimum (Maximum)}, wenn ein \(\delta > 0\) existiert, so dass \(f(x_0)\leq f(x)\) (\(f(x_0)\geq f(x)) \) für \(\lvert x-x_0\rvert < \delta \).\par
	Präziser gesprochen ist \(x_0\) eine lokale \emph{Extremalstelle}\\ (Minimal- oder Maximalstelle), und \(f(x_0)\) ist der lokale \emph{Extremalwert} (Minimal- oder Maximalwert).
\end{karte}

\begin{karte}{Charakterisierung stationäre Punkte}
	Sei \(f'(x_0)=0,\ f^{(k)}(x_0)=0 \text{ für } k=2\dots n \), jedoch \\ \(f^{(n+1)} (x_0) \neq 0\).
	\begin{enumerate}[label=\(\triangleright \)]
		\item Im Fall \((n+1)\) gerade, ist \(x_0\)
		      \begin{enumerate}[label=--]
		      	\item für \( f^{(n+1)}(x_0)>0 \) ist ein \emph{lokales Minimum},
		      	\item für \( f^{(n+1)}(x_0)<0 \) ist ein \emph{lokales Maximum}.
		      \end{enumerate}
		\item Im Fall das \( (n+1) \) ungerade ist \(x_{(0)}\) ein sogenannter \emph{Sattelpunkt}, d.h.\ in jeder Umgebung von \(x_0\) gibt es Punkte mit \(f(x)>f(x_0)\) und mit \(f(x)<f(x_0)\).\vspace{5mm}\par
		      Beachte: Ein Sattelpunkt ist ein Wendepunkt mit \(f'(x_0)=0\).
	\end{enumerate}
\end{karte}
\subsection{Iterationsverfahren}
\begin{karte}{Rekursive Definition des Newton-Verfahrens}
	\begin{align}
		x_{(n+1)}:=x_{n}-\frac{f(x_n)}{f'(x_n)},\ \ n=0,1,2,\dots
	\end{align}
\end{karte}

	\subsection{Riemann Integral}

\begin{karte}{Wie lautet der erste Mittelwertsatz der Integralrechnung}
	Ist \(f(x)\) stetig auf \([a,b]\), dann existiert ein \(\xi\in[a,b]\) mit
	\begin{align}
		\int_{a}^{b}f(x)dx=f(\xi)(b-a) 
	\end{align}
\end{karte}

\begin{karte}{Zweiter Mittelwertsatz der Integralrechnung}
	Sei \(f(x)\) stetig auf \([a,b]\), und \(g:[a,b]\to \mathbb{R}\) sei integrierbar, wobei \(g(x)\geq0,\ x\in[a,b],\ \int_{a}^{b} g(x)dx>0\). Dann existiert ein \(\xi\in[a,b]\) mit
	\begin{align}
		\int_{a}^{b} f(x)g(x)dx=f(\xi)\int_{a}^{b} g(x)dx 
	\end{align}
\end{karte}


\begin{karte}{Hauptsatz der Differential und Integralrechnung}
	\begin{enumerate}[label= (\roman*)]
		
		\item Sei \(f:[a,b] \to \mathbb{R}\) stetig. Dann ist die Funktion \(F(x) = \int_a^x f(\xi)d\xi \) auf \([a,b]\) stetig differenzierbar, und es gilt
		      \begin{align}
		      	F'(x) = \frac{d}{dx} \int_a^x f(\xi)d\xi=f(x) 
		      \end{align}
		      
		\item Sei \(f: I \to \mathbb{R}\) stetig, und \(F\) sei eine Stammfuntkion von \(f\). Dann gilt für \(a,b \in I\)
		      \begin{align}
		      	\int_a^b f(x)dx = F(b) - F(a) =: F(x)|_a^b 
		      \end{align}
	\end{enumerate}
\end{karte}

	\subsection{Funktionenfolgen}
\begin{karte}{Punktweise, gleichmäßige Konvergenz}
	{\normalsize
		\begin{enumerate}[label=\(\triangleright \)]
			\item Existiert \(\displaystyle\lim_{n\to\infty}f_{n}(\xi)\) für \(\xi\in D \), dann heißt \( \{ f_n \} \) an der Stelle \( \xi \) \emph{konvergent}.
			\item Konvergiert \( \{ f_n \} \) an \(x\) für alle \(x\in I \subseteq D\), so heißt \( \{ f_n \} \) \emph{punktweise konvergent in \(I\)}, und dann existiert eine \emph{Grenzfunktion} \(f:I\to\mathbb{R}\) mit
			      \begin{align}
			      	f(x):=\lim_{n\to\infty}f_{n}(x),\ x\in I
			      \end{align}
			\item Die Funktionenfolge \(f_{n}\) heißt \emph{gleichmäßig konvergent} auf \( I \) gegen die Funktion \( f \), wenn
			      \begin{align}
			      	\lim_{n\to\infty} \sup_{x\in I}\, \lvert f_{n}(x)-f_{x}\rvert = 0
			      \end{align}
		\end{enumerate}
		In \glqq\(\varepsilon\)-\(N(\varepsilon)\)-Terminologie \grqq\ ausgedrückt bedeutet dies:
		Zu jedem \(\varepsilon > 0\) existiert ein Index \(N=N(\varepsilon)>0\), so dass für alle \(x\in I\) und für alle \(n\in\mathbb{N}\) mit \(n\geq N(\varepsilon)\) gilt
		\begin{align}
			\lvert f_{n}(x)-f(x)\rvert < \varepsilon
		\end{align}
	}
\end{karte}

\subsection{Potenzreihen, Taylorreihen}
\begin{karte}{Wie lautet das Restglied der Taylorreihe in Integraldarstellung}
	\begin{multline}
		R_{n+1}(x)=\frac{h^{n+1}}{n!}\int_{0}^1{(1-\sigma)}^{n}f^{(n+1)}(x_{0}+\sigma h)d\sigma=\mathcal{O}({\lvert h\rvert}^{n+1})\\ \text{für}\ \ h\ \ \to 0
	\end{multline}
\end{karte}

\begin{karte}{Wie lautet das Taylorpolynom}
	Das Polynom vom Grad \(\leq n \)
	\begin{align}
		T_{n}(x):=\sum_{k=0}^n\frac{f^{(k)}(x_{0})}{k!}{(x-x_{0})}^{k}
	\end{align}
	heißt \emph{\(n\)-tes Taylorpolynom der Funktion} \(f\) zur Entwicklungsstelle \(x_0\), und \(R_{n+1}(x)\) heißt das \emph{Restglied} \((n+1)\)-ter Ordnung.
\end{karte}

\begin{karte}{Taylor-Restglied: Lagrange-Darstellung}
	Sei \(f:[a,b]\to\mathbb{R}\) eine \((n+1)\) mal stetig differenzierbare Funktion. Dann gilt fpr das Taylor-Restglied die Darstellung
{\large
	\begin{align}
		R_{n+1}(x)=\frac{f^{(n+1)}(x_{0}+\vartheta h)}{(n+1)!} h^{n+1},\ h=x-x_0\  \text{für ein}\ \vartheta\in[0,1]
	\end{align}
	}
\end{karte}

\begin{karte}{Taylorentwicklung der Exponentialfunktion, des natürlichen Logarithmus und der trigonometrischen Funktionen}
	\begin{align}
		e^{x}&=\sum_{n=0}^{\infty}\frac{x^{n}}{n!},\ \ x\in\mathbb{R}\\
		\ln{(1-x)}&\sim\sum_{n=1}^{\infty}{(-1)}^{n-1}\:\frac{x^{n}}{n}\\
		\sin{x}=\sum_{n=0}^{\infty}{(-1)}^{n}\frac{x^{2n+1}}{(2n+1)!},&\ \cos{x}=\sum_{n=0}^{\infty}{(-1)}^{n}\frac{x^{2n}}{(2n)!}
	\end{align}

\end{karte}

\begin{karte}{Taylorentwicklung der Hyperbelfunktionen, des Arcustangens und der Binomischen Reihe}
	\begin{align}
		\sinh{x}=\sum_{n=0}^{\infty}\frac{x^{2n+1}}{(2n+1)!},\ \cosh{x}=\sum_{n=0}^{\infty}\frac{x^{2n}}{(2n)!}\\
		\arctan{x}=\sum_{n=0}^{\infty}{(-1)}^{n}\frac{x^{2n+1}}{(2n+1)},\ \text{für}\ x\in [-1,1]\\
		{(1+x)}^{a}=\sum_{n=0}^{\infty}\binom{a}{n}x^{n},\ \text{für}\ \lvert x\rvert <1
	\end{align}
\end{karte}

}
\end{document}
