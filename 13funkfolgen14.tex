\subsection{Funktionenfolgen}
\begin{karte}{Punktweise, gleichmäßige Konvergenz}
	{\normalsize
		\begin{enumerate}[label=\(\triangleright \)]
			\item Existiert \(\displaystyle\lim_{n\to\infty}f_{n}(\xi)\) für \(\xi\in D \), dann heißt \( \{ f_n \} \) an der Stelle \( \xi \) \emph{konvergent}.
			\item Konvergiert \( \{ f_n \} \) an \(x\) für alle \(x\in I \subseteq D\), so heißt \( \{ f_n \} \) \emph{punktweise konvergent in \(I\)}, und dann existiert eine \emph{Grenzfunktion} \(f:I\to\mathbb{R}\) mit
			      \begin{align}
			      	f(x):=\lim_{n\to\infty}f_{n}(x),\ x\in I
			      \end{align}
			\item Die Funktionenfolge \(f_{n}\) heißt \emph{gleichmäßig konvergent} auf \( I \) gegen die Funktion \( f \), wenn
			      \begin{align}
			      	\lim_{n\to\infty} \sup_{x\in I}\, \lvert f_{n}(x)-f_{x}\rvert = 0
			      \end{align}
		\end{enumerate}
		In \glqq\(\varepsilon\)-\(N(\varepsilon)\)-Terminologie \grqq\ ausgedrückt bedeutet dies:
		Zu jedem \(\varepsilon > 0\) existiert ein Index \(N=N(\varepsilon)>0\), so dass für alle \(x\in I\) und für alle \(n\in\mathbb{N}\) mit \(n\geq N(\varepsilon)\) gilt
		\begin{align}
			\lvert f_{n}(x)-f(x)\rvert < \varepsilon
		\end{align}
	}
\end{karte}

\subsection{Potenzreihen, Taylorreihen}
\begin{karte}{Wie lautet das Restglied der Taylorreihe in Integraldarstellung}
	\begin{multline}
		R_{n+1}(x)=\frac{h^{n+1}}{n!}\int_{0}^1{(1-\sigma)}^{n}f^{(n+1)}(x_{0}+\sigma h)d\sigma=\mathcal{O}({\lvert h\rvert}^{n+1})\\ \text{für}\ \ h\ \ \to 0
	\end{multline}
\end{karte}

\begin{karte}{Wie lautet das Taylorpolynom}
	Das Polynom vom Grad \(\leq n \)
	\begin{align}
		T_{n}(x):=\sum_{k=0}^n\frac{f^{(k)}(x_{0})}{k!}{(x-x_{0})}^{k}
	\end{align}
	heißt \emph{\(n\)-tes Taylorpolynom der Funktion} \(f\) zur Entwicklungsstelle \(x_0\), und \(R_{n+1}(x)\) heißt das \emph{Restglied} \((n+1)\)-ter Ordnung.
\end{karte}

\begin{karte}{Taylor-Restglied: Lagrange-Darstellung}
	Sei \(f:[a,b]\to\mathbb{R}\) eine \((n+1)\) mal stetig differenzierbare Funktion. Dann gilt fpr das Taylor-Restglied die Darstellung
{\large
	\begin{align}
		R_{n+1}(x)=\frac{f^{(n+1)}(x_{0}+\vartheta h)}{(n+1)!} h^{n+1},\ h=x-x_0\  \text{für ein}\ \vartheta\in[0,1]
	\end{align}
	}
\end{karte}

\begin{karte}{Taylorentwicklung der Exponentialfunktion, des natürlichen Logarithmus und der trigonometrischen Funktionen}
	\begin{align}
		e^{x}&=\sum_{n=0}^{\infty}\frac{x^{n}}{n!},\ \ x\in\mathbb{R}\\
		\ln{(1-x)}&\sim\sum_{n=1}^{\infty}{(-1)}^{n-1}\:\frac{x^{n}}{n}\\
		\sin{x}=\sum_{n=0}^{\infty}{(-1)}^{n}\frac{x^{2n+1}}{(2n+1)!},&\ \cos{x}=\sum_{n=0}^{\infty}{(-1)}^{n}\frac{x^{2n}}{(2n)!}
	\end{align}

\end{karte}

\begin{karte}{Taylorentwicklung der Hyperbelfunktionen, des Arcustangens und der Binomischen Reihe}
	\begin{align}
		\sinh{x}=\sum_{n=0}^{\infty}\frac{x^{2n+1}}{(2n+1)!},\ \cosh{x}=\sum_{n=0}^{\infty}\frac{x^{2n}}{(2n)!}\\
		\arctan{x}=\sum_{n=0}^{\infty}{(-1)}^{n}\frac{x^{2n+1}}{(2n+1)},\ \text{für}\ x\in [-1,1]\\
		{(1+x)}^{a}=\sum_{n=0}^{\infty}\binom{a}{n}x^{n},\ \text{für}\ \lvert x\rvert <1
	\end{align}
\end{karte}
