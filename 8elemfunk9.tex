\subsection{Elementare Funktionen}



\begin{karte}{(Natürliche und allgemeine) Exponentialfunktion und Logarithmus}
	{\large
		\begin{align}
			e^{x+y}         & =  e^{x}e^{y},    & e^{x\cdot y} & ={(e^{x})}^{y}, & e^{-x}           & = \frac{1}{e^{x}} \\
			\ln{(x\cdot y)} & =\ln{x} + \ln{y}, & \ln{x^{k}}   & =k\ln{x},       & \ln{\frac{1}{x}} & = -\ln{x}         \\
			a^{x+y} &=  a^{x}a^{y},				& a^{x\cdot y}&={(a^{x})}^{y}\\
			&&	a^x&={\left(e^{\ln{a}}\right)}^x=e^{x\ln{a}}\\
			&&\log_{a}x&=\frac{\ln{x}}{\ln{a}},\ 0<x<\infty
		\end{align}
	}
\end{karte}

\begin{karte}{Pythagoräische Identität, Sinussatz, Additionstheoreme}
	\begin{enumerate}[label=\(\triangleright \)]
		\item \(\sin^{2}\alpha+\cos^{2}\alpha = 1\)\dotfill \emph{Pythagoräische Identität}
		\item \(\frac{\sin{\alpha}}{a}=\frac{\sin{\beta}}{b}=\frac{\sin{\gamma}}{c} \)  \dotfill \emph{Sinussatz}
		\item \(c^2=a^2+b^2-2ab\cos{\gamma} \)  \dotfill \emph{Cosinussatz}
		\item \(\sin{(\alpha \pm \beta)} 	=	\sin{\alpha}\cos{\beta}	\pm	\cos{\alpha}\sin{\beta}\)
		\item \(\cos{(\alpha \pm \beta)} 	=	\cos{\alpha}\cos{\beta}	\mp	\sin{\alpha}\sin{\beta}\)
	\end{enumerate}
\end{karte}

\subsection{Differentialrechung}

\begin{karte}{Differenzierbarkeit}
	Sei \(f:(a,b)\to\mathbb{R}\) und \(x\in(a,b)\). Wenn \(\displaystyle\lim_{h\to0}\frac{f(x+h)-f(x)}{h}\) existiert, dann heißt \(f\) \emph{differenzierbar} an der Stelle \(x\). Man schreibt
	\begin{align}
		f'(x):=\lim_{h\to0}\frac{f(x+h)-f(x)}{h}
	\end{align}
	und bezeichnet \(f'(x)\) als die \emph{1. Ableitung} oder \emph{Differentialquotient} von \(f\) an der Stelle \(x\).
\end{karte}

\begin{karte}{Ableitung der Umkehrfunktion}
	Sei \( f(x) \) in \((a,b)\) differenzierbar und streng monoton, weiters sei \(f'(x)\neq 0, x\in(a,b)\). Dann exisitiert die Umkehrfunktion \(f^{-1}\) und ist differenzierbar. Es gilt
	\begin{align}
		(f^{-1}(y))'=\frac{1}{f'(f^{-1}(y))}
	\end{align}
\end{karte}

\begin{karte}{Ableitungen von \(\ln{x}\), \(\log_{a}{x}\) und den zyklometrischen Funktionen}
	\begin{align}
		\frac{d}{dx}\ln{x}     & =\frac{1}{x}\, ,            &   & \frac{d}{dx}\log_a x = \frac{1}{\ln{a}}\frac{1}{x} \\
		\frac{d}{dx}\arcsin{x} & =\frac{1}{\sqrt{1-x^2}}\, , &   & x\in(-1,1)                                         \\
		\frac{d}{dx}\arccos{x} & =-\frac{1}{\sqrt{1-x^2}}\,, &   & x\in(-1,1)                                         \\
		\frac{d}{dx}\arctan{x} & =\frac{1}{1+x^2}\, ,        &   & x\in\mathbb{R}
	\end{align}
\end{karte}

\begin{karte}{Ableitung von \(\arsinh{x}\), \(\arcosh{x}\) und \(\artanh{x}\)}
	\begin{align}
		\frac{d}{dx}\arsinh{x} & =\frac{1}{\sqrt{x^2+1}}\, ,      & x & \in\mathbb{R} \\
		\frac{d}{dx}\arcosh{x} & =\pm\frac{1}{{\sqrt{x^2-1}}}\, , & x & >1            \\
		\frac{d}{dx}\artanh{x} & =\frac{1}{1-x^2}\, ,             & x & \in(-1,1)
	\end{align}
\end{karte}

\begin{karte}{Wie lautet der 1. Mittelwertsatz der Differentialrechnung?}

	Sei \(f(x)\) stetig auf \([a,b]\) und differenzierbar auf \((a,b)\). Dann existiert ein \(\xi \in (a,b)\) mit
	\begin{align}
		f'(\xi)=\frac{f(b)-f(a)}{b-a}
	\end{align}
	{\large
		\emph{Geometrische Interpretation:} Es gibt mindestens einen Punkt \(\xi \in (a,b)\), an dem die Steigung der Tangente an den Graphen von $f$ gleich der Steigung der Geraden durch die Punkte \((a, f(a))\) und \((b, f(b))\) ist. Siehe Abb. 9.4 im Skript.
	}
\end{karte}



\begin{karte}{Wie lautet der 2. Mittelwertsatz der Differentialrechnung?}
	Seien \(f, g\) stetig auf \([a,b]\) und differenzierbar auf \(a,b\). Dann existiert \(\xi\in(a,b)\) mit
	\begin{align}
		f'(\xi)(g(b)-g(a))=g'(\xi)(f(b)-f(a))
	\end{align}
	(Der Speizialfall \(g(x)=x\) ergibt wiederum den 1. Mittelwertsatz.)
\end{karte}



\begin{karte}{Wie lautet der Satz von Rolle?}
	Sei \(f(x)\) stetig auf \([a,b]\), differenzierbar auf \((a,b)\) und gilt \(f(a)=f(b)\). Dann existiert ein \(\xi \in (a,b) \) mit \(f'(\xi)=0\)
\end{karte}

\begin{karte}{Maximalbetrag der Ableitung als Lipschitzkonstante}
	Sei \(f\) stetig differenzierbar auf \([a,b]\). Dann ist \(f\) Lipschitz-stetig auf \([a,b]\), d.h.\ es gilt
	\begin{align}
		\lvert f(x_1)-f(x_2)\rvert \leq L\lvert x_1 - x_2\rvert \  \forall\  x_1,x_2 \in [a,b]
	\end{align}
	mit \(L=\displaystyle\max_{x\in[a,b]}\lvert f'(x)\rvert \) als kleinstmöglicher Lipschitzkonstante.

\end{karte}

\begin{karte}{Regel von de l'Hospital für \(0/0\)}
	Die Funktionen \(f,g:[c,c+\varepsilon]\to\mathbb{R}\) seien stetig und auf \((c,c+\varepsilon)\) differenzierbar. Es gelte
	\begin{align}
		  & f(c)=g(c)=0 & und\ \& g'(x) \neq 0, x \in(c,c+\varepsilon) &                                    &  \\
		\intertext{%
		Falls der Grenzwert}
		& \lim_{x\to c+}\frac{f'(x)}{g'(x)}=\gamma &        &(\gamma=\pm\infty\ \ zugelassen!)  \\
		\intertext{%
		existiert, dann gilt auch}
		& \lim_{x\to c+}\frac{f(x)}{g(x)}=\gamma   &        &
	\end{align}

\end{karte}

\begin{karte}{Wie lautet die Leibnizsche Produktregel}
	\begin{align}
		{(fg)}^{(n)}=\sum_{k=0}^{n} \binom{n}{k}f^{(k)}g^{(n-k)}
	\end{align}
\end{karte}
