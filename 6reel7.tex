\subsection{Reelle Funktionen}
\begin{karte}{Stetigkeit (Limes-Definition)}
	Sei \(A\subseteq\mathbb{R}\) offen. Eine Funktion \(f:D\to\mathbb{R}\) heißt \emph{stetig an der Stelle} \(c \in D\), wenn für jede konvergente Folge \(\{ x_{n} \} \) in \(D\) mit \(\displaystyle\lim_{n\to\infty} x_n = c\) gilt:
	\begin{align}
		\lim_{n\to\infty} f(x_n)=f(c),\ \ kurz:\ \lim_{x\to c}f(x)=f(c)
	\end{align}
\end{karte}

\begin{karte}{Stetigkeit (\(\varepsilon\text{-}\delta \)-Definition)}
	Sei \(A\subseteq\mathbb{R}\) offen. Eine Funktion \(f:D\to\mathbb{R}\) heißt \emph{stetig an der Stelle} \(c \in D\), wenn zu jeder reelen Zahl \(\varepsilon > 0\) eine reele Zahl \(\delta=\delta(\varepsilon)>0\) existiert, so dass für alle x mit \(|c-x|<\delta \) gilt
	\begin{align}
		|f(c)-f(x)|<\varepsilon
	\end{align}

\end{karte}

\begin{karte}{Lipschitz-Stetigkeit einer Funktion \(f\).}%: $\mathbb{R} \to\mathbb{R}$.}
	Eine Funktion \(f\) heißt \emph{Lipschitz-stetig} auf dem Intervall \(I\), wenn es eine Konstante \(L\geq0\) gibt, so dass gilt
	\begin{align}
		\lvert f(x_1) - f(x_2) \rvert\leq L\ \lvert x_1 - x_2 \rvert\  \forall \  x_1,x_2\in I
	\end{align}
	{\large Man beachte, dass die Lipschitzkonstante von dem betrachteten Intervall abhängt. Viele Funktionen (z.\,B. Polynome) sind Lipschitz-stetig auf jedem beschränkten Intervall, aber nicht auf ganz $\mathbb{R}$\par}
	\vspace{5mm}
	\emph{Lipschitz-Stetigkeit} impliziert \emph{gleichmäßige Stetigkeit.}

\end{karte}

\subsection{Rationale Funktionen}

\begin{karte}{Fundamentalsatz der Algebra}
	Jedes Polynom vom Grad \(\geq 1\) hat mindesten eine (reelle oder komplexe) Nullstelle
\end{karte}

\begin{karte}{allgemeine Lösungsformel}
	\begin{align}
		x_{1,2}=\frac{-b\pm\sqrt{b^2-4ac}}{2a}
	\end{align}
\end{karte}
