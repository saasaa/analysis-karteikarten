\subsection{Unendliche Reihen}
\begin{karte}{Geometrische Reihe}
	\begin{align}
		\sum_{k=0}^{\infty} =\frac{1}{1-q}\ \text{für} \ \lvert q\rvert < 1
	\end{align}
\end{karte}

\begin{karte}{Formulieren Sie das Leibnitz-Kriterium für die Konvergenz von alternierenden Reihen.}
	{\large Eine Reihe der Bauart
		\begin{align}
			\sum_{k=0}^{\infty} {(-1)}^{k} a_k, \ mit \ \ a_k > 0\ \text{für }k=0,1,2,\dots
		\end{align}
		heißt \emph{alternierende Reihe.}}\vspace{5mm}\par
	Falls die Folge \( \{ a_k \} \) monoton gegen 0 konvergiert, dann ist die alternierende Reihe
	\begin{align}
		\sum_{k=0}^{\infty}{(-1)}^k a_k
	\end{align}
	konvergent.
\end{karte}

\begin{karte}{Was bedeutet absolute Konvergenz?}
	Eine Reihe \(\sum_{k=1}^{\infty}a_k\) heißt \emph{absolut konvergent}, wenn die Reihe \(\sum_{k=1}^{\infty}\lvert a_k \rvert \) konvergiert. \vspace{5mm}\par
	Jede absolut konvergente Reihe ist konvergent.
\end{karte}

\begin{karte}{Konvergenzkriterien: Majoranten- und Minorantenkriterium}
	\begin{enumerate}[label=\textbullet]

		\item Sei \(\sum_{k=1}^{\infty}a_k\) eine Reihe mit positiven Glieder, also \\ \( a_k > 0\). Falls ein \(M \in\ \mathbb{N}\) existiert, so dass für alle \(k \geq\ M\) gilt \(\lvert c_k\rvert \leq\ a_k\), dann heißt \(\sum_{k=1}^{\infty}a_k\) \emph{Majorante} der Reihe \(\sum_{k=1}^{\infty}c_k\)
		      \begin{enumerate}[label=\(\triangleright \)]
		      	\item Eine Reihe, die eine kovergente Majorante besitzt, ist absolut konvergent.
		      \end{enumerate}
		\item Sei \(\sum_{k=1}^{\infty}b_k\) eine Reihe mit positiven Glieder, also \\ \( b_k > 0\). Falls ein \(M \in \mathbb{N}\) existiert, so dass für alle \(k \geq M\) gilt \(\lvert c_k\rvert \geq a_k\), dann heißt \(\sum_{k=1}^{\infty}b_k\) \emph{Minorante} der Reihe \(\sum_{k=1}^{\infty}c_k\)
		      \begin{enumerate}[label=\(\triangleright \)]
		      	\item Eine Reihe, die eine divergente Minorante besitzt, kann nicht absolut konvergieren.
		      \end{enumerate}

	\end{enumerate}
\end{karte}

\begin{karte}{Quotientenkriterium für absolute Konvergenz einer Reihe}
	Falls für die Reihe \(\sum_{k=1}^{\infty}a_k\) ab einem gewissen Index \(N\), also für alle \( k \geq N\) gilt
	\begin{enumerate}[label=\(\triangleright \)]

		\item \(\lvert \frac{a_{k+1}}{a_k}\rvert \leq q < 1\), dann ist die Reihe absolut konvergent.

		\item \(\lvert \frac{a_{k+1}}{a_k}\rvert \geq  1\), dann ist die Reihe divergent.

		\item \(\lvert \frac{a_{k+1}}{a_k}\rvert \leq  1\), jedoch nicht \(<1\), dann ist keine allgemeine Aussage möglich.
	\end{enumerate}

\end{karte}

\begin{karte}{Quotientenkriterium (Grenzwertformulierung)}
	Falls der Grenzwert
	\begin{align}
		\lim_{k\to\infty} \lvert \frac{a_{k+1}}{a_k}\rvert =: r
	\end{align}
	existiert, dann gilt:

	\begin{enumerate}[label=\(\triangleright \)]

		\item für \(r<1\) ist die Reihe absolut konvergent,
		\item für \(r>1\) ist sie divergent,
		\item für \(r=1\) ist keine Aussage möglich.
	\end{enumerate}


\end{karte}

\begin{karte}{Wurzelkriterium für absolute Konvergenz einer Reihe}
	Falls für eine Reihe \(\sum_{k=1}^{\infty}a_k\) ab einem gewissen Index, also für  alle \(k \geq N\) gilt:

	\begin{enumerate}[label=\(\triangleright \)]

		\item \(\sqrt[k]{\lvert a_k\rvert} \leq q < 1\), dann ist die Reihe absolut konvergent.

		\item \(\sqrt[k]{\lvert a_k\rvert} \geq  1\), dann ist die Reihe divergent.

		\item \(\sqrt[k]{\lvert a_k\rvert} \leq  1\), jedoch nicht \(\leq q <1\), dann ist keine  Aussage möglich.
	\end{enumerate}

\end{karte}

\begin{karte}{Wurzelkriterium (Grenzwertformulierung)}
	Falls der Grenzwert
	\begin{align}
		\lim_{k\to\infty} \sqrt[k]{\lvert a_k\rvert} = r
	\end{align}
	existiert, dann gilt:
	\begin{enumerate}[label=\(\triangleright \)]
		\item für \(r < 1\) ist die Reihe absolut konvergert,
		\item für \(r > 1\) ist sie divergent,
		\item für \(r = 1\) ist keine Aussage möglich.
	\end{enumerate}
\end{karte}

\begin{karte}{Reihendarstellung für \(e\), allgemeine Exponentialreihe}
	Die Eulersche Zahl
	\begin{align}
		e=\lim_{n\to\infty}{\left(1+\frac{1}{n}\right)}^n
	\end{align}
	besitzt die Darstellung als unendliche Reihe:
	\begin{align}
		e=\sum_{k=0}^{\infty}\frac{1}{k!}
	\end{align}
	allgemeine Exponentialreihe:
	\begin{align}
		\lim_{n\to\infty}{\left(1+\frac{x}{n}\right)}^n=\sum_{k=0}^{\infty}\frac{x^n}{k!}=1+x+\frac{x^2}{2!}+\frac{x^3}{3!}+\cdots=:e^x
	\end{align}
\end{karte}
