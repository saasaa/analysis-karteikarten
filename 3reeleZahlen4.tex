\subsection{Die reelen Zahlen}

\begin{karte}{Dreiecksungleichungen}
	Für alle \(x,y \in \mathbb{R}\) gilt.
	\begin{enumerate}[label=\(\triangleright \)]
		\item \(\lvert xy    \rvert =      		\lvert x\rvert    \lvert y\rvert  \)
		\item \(\lvert x\pm y\rvert \leq\  		\lvert x\rvert +  \lvert y\rvert  \)  \dotfill \emph{Dreiecksungleichung}
		\item \(\lvert x\pm y\rvert \geq\ \lvert\lvert x\rvert -  \lvert y\rvert\rvert \)  \dotfill \emph{inverse Dreiecksungleichung}
		      
		      
	\end{enumerate}
\end{karte}

\begin{karte}{Typen von Punken}
	{\large
		Sei \(A \subseteq \mathbb{R}\). Ein Punkt \(x \in \mathbb{R}\) heißt.
		\begin{enumerate}[label=$\triangleright$]
			\item \emph{innerer Punkt} von A, wenn ein \(\varepsilon > 0\) existiert, so dass \(K(x,\varepsilon) 		\subseteq A\);
			\item \emph{äußerer Punkt} bezüglich \(A\), wenn \(x\) innerer Punkt von \(A^c\) ist;
			\item \emph{Randpunkt} von \(A\), wenn  jede \(\varepsilon \)-Umgebung \(K(x,\varepsilon)\) einen Punkt von \(A\) und einen Punkt von \(A^c\) enthält;
			\item \emph{Häufungspunkt} von \(A\), wenn jede \(\varepsilon \)-Umgebung von \( x \) \emph{unendlich viele} Punkte von \(A\) enthält. Gleichbedeutend damit ist, dass jede punktierte \(\varepsilon \)-Umgebung von \(x\) mintestens einen Punkt von \(A\) enthält.
			\item \emph{isolierter Punkt} von \(A\), wenn \(x \in A \) und  wenn es eine punktierte \(\varepsilon \)-Umgebung \(K_{r}(x,\varepsilon)\) gibt, so dass \(K_{r}(x,\varepsilon) \cap A = \emptyset \).
			      
		\end{enumerate}
	}
	
\end{karte}


\begin{karte}{Wie lautet der Satz von Bolzano-Weierstraß?}
	Jede \emph{beschränkte, unendliche} Teilmenge von \(\mathbb{R}\) besitzt mindestens einen Häufungspunkt.
\end{karte}
\subsection{Zahlenfolgen}
