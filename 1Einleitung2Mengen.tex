\subsection{Einleitung}
\begin{karte}{Wie lautet das Prinzip der vollständigen Induktion?}
	Will man eine Aussage \(A(n)\) für jede natürlich Zahl \(n\geq n_{0}\)  beweisen, geht man folgendermaßen vor:
	\begin{enumerate}[label= (\roman*)]
		\item \emph{Induktionsanfang:} Man zeigt die Richtigkeit von \(A(n_{0})\).
		\item \emph{Induktionsschluss:} Man zeigt, dass aus der Richtigkeit von \(A(n)\) für beliebiges \(n\), \(n\geq n_{0}\) (die sogenannte \emph{Induktionsannahme}), die Richtigkeit von \(A(n+1)\) folgt.
	\end{enumerate}
	In konkreten Fällen ist of \(n_{0}=0\) oder \(1\), aber das ist unwesentlich. Das Prinzip ist auch für \(n\in\mathbb{Z}\) anwendbar, d.h. \(n \geq n_{0}\in\mathbb{Z}\)
\end{karte}.

\begin{karte}{Geometrische Summe}
	Für alle reelen Zahlen \(q \in \mathbb{R},\ q\neq1\), und alle \(n \in \mathbb{N}_0\) gilt
	\begin{align}
		\sum_{k=0}^{n} q^k = \frac{1-q^{n+1}}{1-q}
	\end{align}
\end{karte}

\begin{karte}{Binomialkoeffizient}
	Für \(n\) und \(k \in \mathbb{N}_0\), \( k \leq n\) ist der \emph{Binomialkoeffizient} \(\binom{n}{k}\) definiert durch
	\begin{align}
		\binom{n}{k} := \frac{n!}{k!(n-k)!}
	\end{align}
\end{karte}

\begin{karte}{Wie lautet der Binomische Lehrsatz?}
	Für alle x, y \(\in \mathbb{R}\) und \(n \in \mathbb{N}\) gilt:
	\begin{align}
		{(x + y)}^n = \sum_{k=0}^{n}\binom{n}{k} x^{n-k}y^k=\sum_{k=0}^{n}\binom{n}{k} x^{k}y^{n-k}
	\end{align}
\end{karte}

\subsection{Mengen und Abbildungen}


\begin{karte}{Injektivität, Surjektivität, Bijektivität}
	{\large
		Eine Abbildung \(f\) heißt \emph{injektiv}, wenn für alle \(a_1,a_2 \in A\) gilt:
		\begin{align}
			a_1 \neq a_2 \implies f(a_1) \neq f(a_2),
		\end{align}
		d.h.\ zu jedem $b\in B$ gibt es \emph{höchstens} ein $a\in A$ mit\\ $b=f(a)$.

		Eine Abbildung \(f\) heißt \emph{surjektiv}, wenn jedes Element von \(B\)  als Bild eines Elements von $A$ auftritt:
		\begin{align}
			\forall \ b \in B: \exists a\in A \ mit \ b=f(a)
		\end{align}
		d.h.\ zu jedem \(b\in B\) gibt es \emph{mindestens} ein \(a\in A\) mit\\ \(b=f(a)\).

		Eine Abbildung \(f\), die sowohl injektiv als auch surjektiv ist, heißt \emph{bijektiv}. Anders ausgedrückt: Zu jedem \(b \in B\) gibt es \emph{genau ein} \(a \in A\) mit \(b=f(a)\)
	}
\end{karte}
